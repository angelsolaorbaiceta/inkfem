\section{Beam Elements}
\label{sec:beams}

From the mechanics of materials beam theory, the shear stress $V$ in a beam is related to the vertical distributed load $f_y(x)$ as shown by Equation~\ref{eq:shear_vertical_load}:

\begin{equation}
  \label{eq:shear_vertical_load}
  \frac{dV}{dx} = f_y(x)
\end{equation}

Let $v(x)$ be the vertical displacement of the beam's centroidal axis.
Then, the bending moment $M(x)$ can be expressed in terms of this displacement (Equation~\ref{eq:beam_bending_moment}):

\begin{equation}
  \label{eq:beam_bending_moment}
  M(x) = E I \frac{d^2v(x)}{dx^2}
\end{equation}

The relationship between the shear stress $V(x)$ and the bending moment $M(x)$ is given by Equation~\ref{eq:shear_bending_relation}:

\begin{equation}
  \label{eq:shear_bending_relation}
  V(x) = \frac{dM(x)}{dx} = \frac{d}{dx} \left( E I \frac{d^2v(x)}{dx^2} \right)
\end{equation}

And hence, the beam's governing differential equation, provided $EI$ is constant, is given by Equation~\ref{eq:beam_governing_eq}:

\begin{equation}
  \label{eq:beam_governing_eq}
  E I \frac{d^4v(x)}{dx^4} = f_y(x)
\end{equation}


\subsection{Energy Formulation}

Energy...


\subsection{Displacement Field And Interpolation Functions}

The vector $\left\{ q^e \right\}$ contains the beam finite element start and end nodes vertical displacements and rotations:

\[
  \left\{ q^e \right\} =
  \begin{Bmatrix}
    v_1 \\
    \theta_1 \\
    v_2 \\
    \theta_2 \\
  \end{Bmatrix} =
  \begin{Bmatrix}
    v_1 \\
    \left( \frac{dv}{dx} \right)_1 \\
    v_2 \\
    \left( \frac{dv}{dx} \right)_2 \\
  \end{Bmatrix}
\]

Let $v(x)$ be the vertical displacements field in the beam finite element of length $L$.
We can assume a cubic displacement field inside the element, as described by Equation~\ref{eq:beam_disp_field}:

\begin{equation}
  \label{eq:beam_disp_field}
  v(x) = N_1 v_1 + N_2 \theta_1 + N_3 v_2 + N_4 \theta_2
\end{equation}

where:

\begin{equation}
  \begin{split}
    N_1(x) & = \frac{1}{L^3} \left( 2x^3 -3x^2L + L^3 \right) \\
    N_2(x) & = \frac{1}{L^2} \left( x^3 - 2x^2L + xL^2 \right) \\
    N_3(x) & = \frac{1}{L^3} \left( -2x^3 - 3x^2L \right) \\
    N_4(x) & = \frac{1}{L^2} \left( x^3 - x^2L \right) \\
  \end{split}
\end{equation}

These N interpolation functions fulfill the following conditions:

\[
  \begin{aligned}
    N_1(0) & = 1    &    N_1'(0) & = 0    &    N_1(L) & = 0    &    N_1'(L) & = 0 \\
    N_2(0) & = 0    &    N_2'(0) & = 1    &    N_2(L) & = 0    &    N_2'(L) & = 0 \\
    N_3(0) & = 0    &    N_3'(0) & = 0    &    N_3(L) & = 1    &    N_3'(L) & = 0 \\
    N_4(0) & = 0    &    N_4'(0) & = 0    &    N_4(L) & = 0    &    N_4'(L) & = 1 \\
  \end{aligned}
\]

The displacement field can be rewritten as:

\[
  v(x) = \left[ N \right] \left\{ q^e \right\} =
  \begin{bmatrix}
    N_1 & N_2 & N_3 & N_4 \\
  \end{bmatrix}
  \begin{Bmatrix}
    v_1 \\
    \theta_1 \\
    v_2 \\
    \theta_2 \\
  \end{Bmatrix}
\]


\subsection{Distributed Loads}

loads...
