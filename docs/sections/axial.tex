\section{Axial Elements}

An \emph{axial element} is a linear resistant element whose deformation happens in the direction or its directrix only.
The linear elements used by \emph{InkFEM} are a combination of axial and beam elements (see Section~\ref{sec:beams}).


\subsection{Energy Formulation}

The total energy of an axial element\dots

\subsection{Displacement Field And Interpolation Functions}

Let $\vec{u}(x)$ be the displacements field in the axial finite element of length $L$.
$u_1$ is the displacement of the element's node 1 in the X direction, and $u_2$ the displacement of node 2 in the X direction.
The displacements field can be chosen to vary linearly inside the finite element, so it can be interpolated from the values of $u_1$ and $u_2$ like so:

\begin{equation}
  \vec{u}(x) = N_1 u_1 + N_2 u_2
\end{equation}

where:

\begin{equation}
  \begin{split}
    N_1(x) = 1 - \frac{x}{L} \\
    N_2(x) = \frac{x}{L} \\
  \end{split}
\end{equation}


\subsection{Distributed Loads}

We consider distributed axial loads that vary linearly with respect to the X direction:

\begin{equation}
  q_x(x) = a + bx
\end{equation}

The work done by these forces can be obtained by the following integration:

\[
  W_{q_x} = \int_{0}^{L} q_x \vec{u}(x) dx = \int_{0}^{L} (a + bx) \left[ \left( 1 - \frac{x}{L} \right) u_1 + \left( \frac{x}{L} \right) u_2 \right] dx
\]

Which yields a result of:

\[
  W_{q_x} =
  \begin{bmatrix}
    \frac{a L}{2} + \frac{b L^2}{6} & \frac{a L}{2} + \frac{b L^2}{3} \\
  \end{bmatrix}
  \begin{bmatrix}
    u_1 \\
    u_2 \\
  \end{bmatrix}
\]

Therefore, a linear axial load $q_x(x)$ can be distributed over the two nodes in the finite element adding the forces:

\begin{equation}
  \begin{split}
    F_x^1 = \frac{a L}{2} + \frac{b L^2}{6} \\
    F_x^2 = \frac{a L}{2} + \frac{b L^2}{3} \\
  \end{split}
\end{equation}
